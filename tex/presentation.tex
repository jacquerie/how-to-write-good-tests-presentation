\documentclass[10pt]{beamer}

\usepackage{amssymb}
\usepackage[english]{babel}
\usepackage[utf8]{inputenc}
\usepackage{minted}

\usetheme{Copenhagen}
\setbeamertemplate{navigation symbols}{}

\title{How To Write Good Tests}
\author{
  Jacopo Notarstefano,\\
  \texttt{jacopo.notarstefano [at] cern.ch}
}
\date{May 16, 2017}

\begin{document}
  \begin{frame}[plain]
    \titlepage{}
  \end{frame}

  \begin{frame}{Google Testing On The Toilet}
    Google is a fairly successful company in the software world. If you haven't
    heard of them, you should probably google their name.

    \vspace{0.25cm}

    They have an interesting engineering practice:

    \vspace{0.25cm}

    \begin{block}{}
      ``We write flyers about everything from dependency injection to code
      coverage, and then regularly plaster the bathrooms all over Google with
      each episode, almost 500 stalls worldwide.''
    \end{block}

    \vspace{0.5cm}

    \url{https://testing.googleblog.com/2007/01/introducing-testing-on-toilet.html}
  \end{frame}

  \begin{frame}{Why Should I Write Tests?}
    Unlike the code that we write for ourselves or for a school assignment, code
    that we write for INSPIRE is going to be refactored, fixed or expanded.

    \vspace{0.5cm}

    By writing tests we can ensure that the next programmer that will touch our
    code will be able to build on top of our understanding of the problem.

    \vspace{0.5cm}

    In order for this to happen, our tests should be \emph{good}. Let's see
    what this means.
  \end{frame}

  \begin{frame}{What Is A Good Test?}
    A good test maximizes three qualities:

    \vspace{0.25cm}

    \begin{itemize}
      \item \textbf{Fidelity}: sensitivity to defects in the code under test.
      \item \textbf{Resilience}: indifference to non-breaking changes.
      \item \textbf{Precision}.
    \end{itemize}

    \vspace{0.5cm}

    \url{https://testing.googleblog.com/2014/05/testing-on-toilet-effective-testing.html}
  \end{frame}

  \begin{frame}{Test Behavior, Not Implementation}
    \url{https://testing.googleblog.com/2013/08/testing-on-toilet-test-behavior-not.html}
  \end{frame}

  \begin{frame}{Prefer Testing Public APIs}
    \url{https://testing.googleblog.com/2015/01/testing-on-toilet-prefer-testing-public.html}
  \end{frame}

  \begin{frame}{An Example}
    How would you test this sorting algorithm?

    \vspace{0.25cm}

    \inputminted[fontsize=\scriptsize]{python}{tex/src/selection.py}
  \end{frame}

  \begin{frame}{Change-Detector Tests Considered Harmful}
    \url{https://testing.googleblog.com/2015/01/testing-on-toilet-change-detector-tests.html}
  \end{frame}

  \begin{frame}{Test Behaviors, Not Methods}
    \url{https://testing.googleblog.com/2014/04/testing-on-toilet-test-behaviors-not.html}
  \end{frame}

  \begin{frame}{Naming Unit Tests Responsibly}
    \url{https://testing.googleblog.com/2007/02/tott-naming-unit-tests-responsibly.html}
  \end{frame}

  \begin{frame}{Too Many Tests}
    \url{https://testing.googleblog.com/2008/02/in-movie-amadeus-austrian-emperor.html}
  \end{frame}

  \begin{frame}{Understanding Your Coverage Data}
    \url{https://testing.googleblog.com/2008/03/tott-understanding-your-coverage-data.html}
  \end{frame}

  \begin{frame}{Test Sizes}
    Tests come in different sizes:

    \vspace{0.25cm}

    \begin{table}[H]
      \begin{tabular}{llll}
        \textbf{Feature} & \textbf{Small} & \textbf{Medium} & \textbf{Large} \\
        \hline
        Database & \(\times\) & \checkmark & \checkmark \\
        Network Access & \(\times\) & \texttt{localhost} & \checkmark \\
        \ldots & \ldots & \ldots & \ldots \\
        File System & \texttt{idempotent} & \checkmark & \checkmark \\
        \hline
      \end{tabular}
    \end{table}

    \vspace{0.5cm}

    \url{https://testing.googleblog.com/2010/12/test-sizes.html}
  \end{frame}
\end{document}
